\documentclass[a4paper,11pt, landscape]{article}

% Package pour la présentation (police, colonnes, marges ...)
\usepackage{ae,lmodern}
\usepackage[utf8]{inputenc}
\usepackage{multicol}
\usepackage[top=0.3cm, bottom=0.3cm, left=0.4cm, right=0.4cm]{geometry}
\usepackage{enumitem}

% Package pour les maths
\usepackage{amsthm}
\usepackage{amsmath}
\usepackage{amssymb}
\usepackage{mathrsfs}
\usepackage{amsfonts}
%\usepackage{stmaryrd}

\begin{document}
\begin{multicols*}{3}
\setlength{\columnsep}{1cm}
% *****************
%  Sujet numéro 1
% ***************** 
\centerline{Sujet A - HA 5}
\begin{flushleft}
% *****************
%  Cours
% ***************** 
  \textbf{Partie cours} 
\end{flushleft} 

\begin{enumerate}[leftmargin=*]
  \item Donner la définition de l'ensemble image de $f : E \rightarrow F$.
  \item Enoncer puis démontrer la factorisation de $a^n - b^n$.
\end{enumerate}
% *****************
%  Exercices
% ***************** 
\textbf{Partie exercices}
\begin{enumerate}[leftmargin=*]
  \item Soit $f : x \rightarrow \ln \left(x\right) - 2\sqrt{x}$
 \begin{enumerate}
   \item Étudiez $f$ et en déduire que $\forall x \in \mathbb{R}_+^*, \; \frac{\ln (x)}{\sqrt{x}} < 2$
   \item En déduire la limite de $\frac{\ln(x)}{x}$ quand $x$ tend vers $+\infty$
   \item En déduire la limite de $x\ln(x)$ lorsque $x$ tend vers 0
 \end{enumerate}
  \item Étudier les limites ci-dessous (existence, valeur éventuelle):
  \begin{enumerate}
    \item $\lim\limits_{x \rightarrow -1}\left(\frac{1}{1-x} - \frac{2}{1-x^2}\right)$
    \item $\lim\limits_{x \rightarrow +\infty }\left(\sqrt{x^2+1} - x\right)$
  \end{enumerate}
  \item Calculer la valeur des sommes suivantes:
  \begin{enumerate}
    \item $\sum\limits_{k=1}^n(k \times  k!)$
    \item $\sum\limits_{1\leq i \leq j \leq n}i2^j$
    \item $\sum\limits_{1\leq i \leq j \leq n}i(j-1)$
    \item $\sum\limits_{1\leq p,q \leq n}(p+q)^2$ On peut poser $k = p+q$
  \end{enumerate}
  \item Mettre sous la forme $a + ib$ où $(a,b) \in \mathbb{R}^2$, les nombres complexes suivants:
  \begin{enumerate}
    \item $\frac{3+6i}{3-4i}$
    \item $\left(\frac{1+i}{2-i}\right)^2 + \frac{3+6i}{3-4i}$
    \item $\frac{2+5i}{1-i} + \frac{2-5i}{1+i}$
  \end{enumerate}
\end{enumerate}
\centerline{$\mathcal{MR}$}

\vfill\null
\columnbreak
% *****************
%  Sujet numéro 2
% *****************
\centerline{Sujet B - HA 5}

\begin{flushleft}
% *****************
%  Cours
% ***************** 
  \textbf{Partie cours} 
\end{flushleft} 
\begin{enumerate}[leftmargin=*]
  \item Donner la définition de la factorielle de $n$.
  \item Enoncer puis démontrer la relation de Pascal.
\end{enumerate}
% *****************
%  Exercices
% ***************** 
\textbf{Partie exercices}
\begin{enumerate}[leftmargin=*]
%------------
% Exercice 1
%------------
  \item Trouver toutes les fonctions $f : \mathbb{R} \rightarrow \mathbb{R}$ qui vérifient l'équation fonctionnelle: $\forall x \in \mathbb{R}, \forall y \in \mathbb{R}, \; |f(x) - f(y)| = |x - y|$
%------------
% Exercice 2
%------------
  \item Soit $f \rightarrow ln\left(\sqrt{1 + x^2} - x\right)$. Déterminer le domaine de définition de $f$, puis étudiez sa parité et ses variations.
%------------
% Exercice 3
%------------
\item Calculer la valeur des sommes suivantes:
\begin{enumerate}
  \item $\sum\limits_{k=2}^{100}\frac{1}{k^2-1}$
  \item $\sum\limits_{k=1}^{n}\ln\left(1+\frac{1}{k}\right)$
  \item $\sum\limits_{1\leq i \leq j \leq n}ij$
  \item $\sum\limits_{1\leq i \leq j \leq n}(n-i)(n-j)$
\end{enumerate}  
%------------
% Exercice 4
%------------
\item Mettre sous la forme $a + ib$ où $(a,b) \in \mathbb{R}^2$, les nombres complexes suivants:
\begin{enumerate}
  \item $\frac{5+2i}{1-2i}$
  \item $\left(\frac{-1}{2} + i\frac{\sqrt{3}}{2}\right)^3$
  \item $\frac{(1+i)^9}{(1-i)^7}$
\end{enumerate}
\end{enumerate}
\centerline{$\mathcal{MR}$}
\vfill\null
\columnbreak
% *****************
%  Sujet numéro 3
% ***************** 
\centerline{Sujet C - HA 5}
\begin{flushleft}
% *****************
%  Cours
% ***************** 
  \textbf{Partie cours} 
\end{flushleft} 
\begin{enumerate}[leftmargin=*]
  \item Donner la définition du conjugué de $z = a + ib \in \mathbb{C}$.
  \item Enoncer puis démontrer la formule du Binôme de Newton.
\end{enumerate}
% *****************
%  Exercices
% ***************** 
\textbf{Partie exercices}
\begin{enumerate}[leftmargin=*]
%------------
% Exercice 1
%------------
  \item Soit $f : \mathbb{R} \rightarrow \mathbb{R}$. On suppose $f$ dérivable. Montrer que si $f$ est paire alors $f'$ est impaire. Que peut-on dire si $f$ est impaire? Prouver votre affiramation.
%----------
% Exercice 2
%-----------
  \item Étudier les limites ci-dessous (existence, valeur éventuelle):
  \begin{enumerate}
    \item $\lim\limits_{x \rightarrow 1}\left(\frac{1}{1-x} - \frac{2}{1-x^2}\right)$
    \item $\lim\limits_{x \rightarrow 2}\left(\frac{x^2 - 4}{x^2 - 3x+2}\right)$
    \item $\lim\limits_{x \rightarrow 0 }\left(\sqrt{1+\frac{1}{x}} - \sqrt{\frac{1}{x}}\right)$
    \item $\lim\limits_{x \rightarrow +\infty }\left(\frac{x^2 + |x|}{x}\right)$
  \end{enumerate}
%-----------
% Exercice 3
%-----------
  \item Calculer la valeurs des sommes suivantes:
  \begin{enumerate}
    \item $\sum\limits_{k=1}^n\sum\limits_{j=k}^n\frac{1}{j}$
    \item $\sum\limits_{1\leq i \leq j \leq n}\frac{i}{j}$
    \item $\sum\limits_{1\leq i \leq j \leq n}(i-1)j$
    \item $\sum\limits_{i=1}^ni2^i$
  \end{enumerate}
%-----------
% Exercice 4
%-----------
\item Mettre sous la forme $a + ib$ où $(a,b) \in \mathbb{R}^2$, les nombres complexes suivants:
\begin{enumerate}
  \item $\frac{-2}{1-i\sqrt{3}}$
  \item $\frac{1}{(1+2i)(3-i)}$
  \item $\frac{1+2i}{1-2i}$
  \item $\frac{2+5i}{1-i} + \frac{2-6i}{1+i}$
\end{enumerate}
\end{enumerate}
\centerline{$\mathcal{MR}$}


\end{multicols*}
\end{document}
