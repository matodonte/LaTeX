\documentclass[a4paper,11pt, landscape]{article}

% Package pour la présentation (police, colonnes, marges ...)
\usepackage{ae,lmodern}
\usepackage[utf8]{inputenc}
\usepackage{multicol}
\usepackage[top=0.3cm, bottom=0.3cm, left=0.4cm, right=0.4cm]{geometry}
\usepackage{enumitem}

% Package pour les maths
\usepackage{amsthm}
\usepackage{amsmath}
\usepackage{amssymb}
\usepackage{mathrsfs}
\usepackage{amsfonts}
%\usepackage{stmaryrd}

\begin{document}
\begin{multicols*}{4}
\setlength{\columnsep}{1cm}
% *****************
%  Sujet numéro 1
% ***************** 
\centerline{Sujet A - HA 4}
\begin{flushleft}
% *****************
%  Cours
% ***************** 
  \textbf{Partie cours} 
\end{flushleft} 

\begin{enumerate}[leftmargin=*]
  \item Donner la définition d'un majorant d'un ensemble A. 
  \item Démontrer l'inégalité triangulaire suivante: $\forall x \in \mathbb{R}$, $\forall y \in \mathbb{R}, \; |x + y | \leq |x| + |y|$.
\end{enumerate}
% *****************
%  Exercices
% ***************** 
\textbf{Partie exercices}
\begin{enumerate}[leftmargin=*]
  \item Montrer par contraposition les assertions suivantes:
  \begin{enumerate}
    \item $\forall A, B \in \mathcal{P}(E) \; (A \cap B = A \cup B) \Rightarrow A = B$
    \item $\forall A, B, C \in \mathcal{P}(E) \; (A \cap B = A \cap C \; et \; A \cup B  = A \cup C) \Rightarrow B = C$
  \end{enumerate}
  \item Montrer que: $\forall (x;y) \in (\mathbb{R}_+)^2, \; \sqrt{x+y} < \sqrt{x} + \sqrt{y}$
  \item Soit $f : \mathbb{R} \rightarrow \mathbb{R}$ telle que $f(x) = max\{x, 0\}$.
  \begin{enumerate}
    \item Montrer que $\forall x \in \mathbb{R}, x \leq f(x)$
    \item Montrer que $f$ est croissante.
    \item Montrer que $\forall x \in \mathbb{R}, \; f(x) = f(f(x))$
    \item On pose $F:= \{x \in \mathbb{R} \; | \; x = f(x)\}$ et, pour tout réel x, $F_x := \{y \in F \; | \; x \leq y\}$. Déterminer $F$ et $F_x$ et montrer que $F_x$ a un ppe.
  \end{enumerate}
  \item Étudier les limites ci-dessous (existence, valeur éventuelle):
  \begin{enumerate}
    \item $\lim\limits_{x \rightarrow -1}\left(\frac{1}{1-x} - \frac{2}{1-x^2}\right)$
    \item $\lim\limits_{x \rightarrow +\infty }\left(\sqrt{x^2+1} - x\right)$
  \end{enumerate}
\end{enumerate}
\centerline{$\mathcal{MR}$}

\vfill\null
\columnbreak
% *****************
%  Sujet numéro 2
% *****************
\centerline{Sujet B - HA 4}

\begin{flushleft}
% *****************
%  Cours
% ***************** 
  \textbf{Partie cours} 
\end{flushleft} 
\begin{enumerate}[leftmargin=*]
  \item Donner la définition du plus petit élément d'un ensemble A.
  \item Donner une CNS sur une application $f : \mathbb{R} \rightarrow \mathbb{R}$ pour qu’elle soit croissante et décroissante.
\end{enumerate}
% *****************
%  Exercices
% ***************** 
\textbf{Partie exercices}
\begin{enumerate}[leftmargin=*]
  \item Montrer que $\forall \epsilon > 0 \;\; \exists N \in \mathbb{N}$ tel que $(n \geq N \Rightarrow 2 - \epsilon < \frac{2n+1}{n+2} < 2+\epsilon)$
  \item Soient A et B des parties d'un ensemble E. Montrer que $A \cap B = A \cap C \Leftrightarrow A \cap \complement{B} = A \cap \complement{C}$
  \item Représenter les ensembles suivants:
  \begin{enumerate}
    \item $A:= \{(x,y) \in \mathbb{R}^2 \; | \; |x| + |y| \leq 1 \}$
    \item $B:= \{(x,y) \in \mathbb{R}^2 \; | \; max\{|x|; |y|\} \leq 1 \}$
    \item $C:= \{(x,y) \in \mathbb{R}^2 \; | \; 1 \leq x^2 + y^2 < 2 \}$
    \item $D:= \{(x,y) \in \mathbb{R}^2 \; | \; |x-y| \leq 1\}$
  \end{enumerate}
  \item Soit $f \rightarrow ln\left(\sqrt{1 + x^2} - x\right)$. Déterminer le domaine de définition de $f$, puis étudiez sa parité et ses variations.
  
\end{enumerate}
\centerline{$\mathcal{MR}$}
\vfill\null
\columnbreak
% *****************
%  Sujet numéro 3
% ***************** 
\centerline{Sujet C - HA 4}
\begin{flushleft}
% *****************
%  Cours
% ***************** 
  \textbf{Partie cours} 
\end{flushleft} 
\begin{enumerate}[leftmargin=*]
  \item Donner la définition du graphe d'une fonction $f : E \rightarrow F$.
  \item Montrer que l'ensemble des majorants de $]0; 1[$ est $[1; +\infty[$
\end{enumerate}
% *****************
%  Exercices
% ***************** 
\textbf{Partie exercices}
\begin{enumerate}[leftmargin=*]
  \item Soient A et B des parties de E. On note $A \Delta B = (A \cap \complement_EB) \cup (B \cap \complement_EA)$ la différence symétrique. 
  Montrer que: 
  \begin{enumerate}
    \item $(A \Delta B = A \cap B) \Leftrightarrow (A = B = \emptyset)$
    \item $A \Delta B = B \Delta A$
    \item $A \Delta B = \emptyset \Leftrightarrow A = B$
  \end{enumerate}
  \item Soient $u$ et $v$ deux applications de $\mathbb{R}$ dans $\mathbb{R}$. Soit $D$ une partie de $\mathbb{R}$. 
  \begin{enumerate}
    \item Donner une condition suffisante pour que: $h : D \rightarrow \mathbb{R}$ telle que $h(x) = u(x)^{v(x)}$ soit bien définie. 
    \item On se place dans ce cas. Sur quel domaine $h$ est-elle dérivable? donner une expression de $h'(x)$.
  \end{enumerate}
  \item Soit $f : \mathbb{R} \rightarrow \mathbb{R}$. On suppose $f$ dérivable. Montrer que si $f$ est paire alors $f'$ est impaire. Que peut-on dire si $f$ est impaire? Prouver votre affiramation.
  \item Étudier les limites ci-dessous (existence, valeur éventuelle):
  \begin{enumerate}
    \item $\lim\limits_{x \rightarrow 1}\left(\frac{1}{1-x} - \frac{2}{1-x^2}\right)$
    \item $\lim\limits_{x \rightarrow 2}\left(\frac{x^2 - 4}{x^2 - 3x+2}\right)$
    \item $\lim\limits_{x \rightarrow 0 }\left(\sqrt{1+\frac{1}{x}} - \sqrt{\frac{1}{x}}\right)$
    \item $\lim\limits_{x \rightarrow +\infty }\left(\frac{x^2 + |x|}{x}\right)$
  \end{enumerate}
\end{enumerate}
\centerline{$\mathcal{MR}$}
\vfill\null
\columnbreak
% *****************
%  Sujet numéro 4
% ***************** 
\centerline{Sujet D - HA 4}
\begin{flushleft}
% *****************
%  Cours
% ***************** 
  \textbf{Partie cours} 
\end{flushleft} 
\begin{enumerate}[leftmargin=*]
  \item Donner la définition de l'ensemble image de $f : E \rightarrow F$.
  \item Montrer que la composée de deux fonctions monotones est monotone.
\end{enumerate}
% *****************
%  Exercices
% ***************** 
\textbf{Partie exercices}
\begin{enumerate}[leftmargin=*]
 \item Soient $A, B \subset E$. Résoudre les équations d'inconnue $X \subset E$
 \begin{enumerate}
   \item $A \cup X = B$
   \item $A \cap X = B$
 \end{enumerate}
 \item Soit $f: E \rightarrow E$. Pour $n \in \mathbb{N}*$, on note: $f^n = f \circ f \circ ... \circ f$, et $f^0 = id_E$. Soit $A \subset E$, $A_n = f^n<A>$ et $B = \bigcup_{n \in \mathbb{N}}A_n$.
 Montrer que: 
 \begin{enumerate}
   \item $f<B> \subset B$.
   \item $B$ est la plus petite partie de E stable par $f$ et contenant A
 \end{enumerate}
 \item Soit $f : x \rightarrow \ln \left(x\right) - 2\sqrt{x}$
 \begin{enumerate}
   \item Étudiez $f$ et en déduire que $\forall x \in \mathbb{R}_+^*, \; \frac{\ln (x)}{\sqrt{x}} < 2$
   \item En déduire la limite de $\frac{\ln(x)}{x}$ quand $x$ tend vers $+\infty$
   \item En déduire la limite de $x\ln(x)$ lorsque $x$ tend vers 0
 \end{enumerate}
 \item Trouver toutes les fonctions $f : \mathbb{R} \rightarrow \mathbb{R}$ qui vérifient l'équation fonctionnelle: $\forall x \in \mathbb{R}, \forall y \in \mathbb{R}, \; |f(x) - f(y)| = |x - y|$
\end{enumerate}
\centerline{$\mathcal{MR}$}

\end{multicols*}
\end{document}
