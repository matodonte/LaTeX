\documentclass[a4paper,11pt, landscape]{article}

% Package pour la présentation (police, colonnes, marges ...)
\usepackage{ae,lmodern}
\usepackage[utf8]{inputenc}
\usepackage{multicol}
\usepackage[top=0.3cm, bottom=0.3cm, left=0.3cm, right=0.3cm]{geometry}

% Package pour les maths
\usepackage{amsthm}
\usepackage{amsmath}
\usepackage{amssymb}
\usepackage{mathrsfs}
\usepackage{amsfonts}
%\usepackage{stmaryrd}

\begin{document}
\begin{multicols*}{3}
\setlength{\columnsep}{1cm}
% *****************
%  Sujet numéro 1
% ***************** 
\centerline{Sujet A - HA 2}
\begin{flushleft}
  \textbf{Partie cours} 
\end{flushleft} 
\begin{enumerate}
  \item Donner la définition de la racine carré d'un nombre réel.
  \item Montrer que si A et B sont deux ensembles de E, alors $A \subset B \Rightarrow \complement_E{B} \subset \complement_E{A}$.
\end{enumerate}
\textbf{Partie exercices}
\begin{enumerate}
  \item Montrer que: $A \Rightarrow B$ si et seulement si  $non (B) \Rightarrow non(A)$.
  \item Résolvez les équations/inéquations suivantes. Soit x un nombre réel.
  \begin{enumerate}
    \item $x+1 = \sqrt{x+2}$
    \item $\sqrt{x^2 - 6x + 9} < 4$
    \item $|2x+3| - |x-1| < 3$
    \item $x+a < \sqrt{x+a}\;$ où $a \in \mathbb{R}$
    \item $x+|x-1| = 1 + |x|$
  \end{enumerate}
  \item Dire si les assertions suivantes sont vraies ou fausses et donner leur négation:
  \begin{enumerate}
    \item $\forall x \in \mathbb{R}, \; \exists n \in \mathbb{N} \; | \;x \leq n$
    \item $\exists M \in \mathbb{R_+^*} \; | \; \forall n \in \mathbb{N}, \;|U_n| \leq M$
    \item $\forall x \in \mathbb{R}, \; \forall y \in \mathbb{R}, \; xy = yx$
    \item $\forall x \in \mathbb{R}, \; \exists y \in \mathbb{R} \; | \; yxy^{-1} = x$
  \end{enumerate}
  \item Montrer par contraposition les assertions suivantes:
  \begin{enumerate}
    \item $\forall A, B \in \mathcal{P}(E) \; (A \cap B = A \cup B) \Rightarrow A = B$
    \item $\forall A, B, C \in \mathcal{P}(E) \; (A \cap B = A \cap C \; et \; A \cup B  = A \cup C) \Rightarrow B = C$
  \end{enumerate}
\end{enumerate}
\centerline{$\mathcal{MR}$}

\vfill\null
\columnbreak
% *****************
%  Sujet numéro 2
% *****************
\centerline{Sujet B - HA 2}

\begin{flushleft}
  \textbf{Partie cours} 
\end{flushleft} 
\begin{enumerate}
  \item Donner la définition d'une fonction croissante.
  \item Montrer pour tout $n \in \mathbb{N}, 2^n > n$.
\end{enumerate}
\textbf{Partie exercices}
\begin{enumerate}
  \item Écrire la négation des assertions suivantes où P, Q, R, S sont des propositions:
  \begin{enumerate}
    \item $P \Rightarrow Q$
    \item P ET NON Q
    \item P ET (Q ET R)
    \item P OU (Q ET R)
    \item (P ET Q) $\Rightarrow (R \Rightarrow S)$ 
  \end{enumerate}
  \item Soit f la fonction de $\mathbb{R}$ dans $\mathbb{R}$ définie par: $\forall x \in \mathbb{R}, \; f(x) = x^2+x+1$
  \begin{enumerate}
    \item Montrer que $\exists x \in \mathbb{R}, \; f(x) > 2$
    \item Est-il vrai que $\forall x \in \mathbb{R}, \; f(x) > 2$ ?
    \item Est-il vrai que $\forall x \in \mathbb{R}, \; f(x) > 0$ ?
    \item Résoudre dans $\mathbb{C}, \; f(x) = 0$ sans utiliser $\Delta$ 
  \end{enumerate}
  \item Déterminer m pour que l'équation suivante ait deux racines réelles positives: $m^2x^2 + (m-2)x + 6 = 0$
  \item Montrer que $A \cap B = A \cap C \Leftrightarrow A \cap \complement{B} = A \cap \complement{C}$
\end{enumerate}
\centerline{$\mathcal{MR}$}

\vfill\null
\columnbreak
% *****************
%  Sujet numéro 3
% *****************
\centerline{Sujet C - HA 2}
\begin{flushleft}
  \textbf{Partie cours} 
\end{flushleft} 
\begin{enumerate}
  \item Donner la définition d'un nombre impaire .
  \item Démontrer que $\sqrt{2}$ est irrationnel.
\end{enumerate}
\textbf{Partie exercices}
\begin{enumerate}
  \item Soit m et p deux paramètres réels fixés. Résolvez les équations et innéquations suivantes en x (réel). Pensez à bien déterminer l'ensemble solution:
  \begin{enumerate}
    \item $(2x+m)(x-6p) > 0$
    \item $(mx-\alpha)(4x-p) = 0$ où $\alpha > 0$
    \item $(x+m)^2 = 2(x+p)$
    \item $x+m = \sqrt{x+p} + 4$
    \item $|x-\frac{5}{3}| < m$
  \end{enumerate}
  \item Soit A, B deux ensembles de E, montrer que $\complement_E{(A \cup B)} = \complement_E{A} \cap \complement_E{B}$ et $\complement_E{(A \cap B)} = \complement_E{A} \cup \complement_E{B}$
  \item Montrer que $\forall \epsilon > 0 \;\; \exists N \in \mathbb{N}$ tel que $(n \geq N \Rightarrow 2 - \epsilon < \frac{2n+1}{n+2} < 2+\epsilon)$
  \item Écrire la négation des phrases suivantes:
  \begin{enumerate}
    \item $\forall x \in \mathbb{R} \; \exists n \in  \mathbb{N} \; | \; x \leq n$
    \item $\forall x \in \mathbb{R} \; \forall y \in \mathbb{R} \; xy = yx$
    \item $\forall \epsilon > 0, \; \exists N \in \mathbb{N}, \; \forall n \leq N, \; |U_n| < \epsilon$
    \item $\forall x \in \mathbb{R}, \; \forall \epsilon > 0, \; \exists \alpha > 0 \; | \; \forall f \in \mathcal{F}(E,F), \; \forall y \in \mathbb{R}, \; |x-y|< \alpha \Rightarrow |f(x) - f(y)| < \epsilon$
  \end{enumerate}
\end{enumerate}
\centerline{$\mathcal{MR}$}


\end{multicols*}
\end{document}
