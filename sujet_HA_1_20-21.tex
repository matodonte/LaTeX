\documentclass[a4paper,12pt, landscape]{article}

% Package pour la présentation (police, colonnes, marges ...)
\usepackage{ae,lmodern}
\usepackage[utf8]{inputenc}
\usepackage{multicol}
\usepackage[top=0.3cm, bottom=0.3cm, left=0.3cm, right=0.3cm]{geometry}

% Package pour les maths
\usepackage{amsthm}
\usepackage{amsmath}
\usepackage{amssymb}
\usepackage{mathrsfs}
\usepackage{amsfonts}
%\usepackage{stmaryrd}

\begin{document}
\begin{multicols*}{3}
\setlength{\columnsep}{1cm}
% *****************
%  Sujet numéro 1
% ***************** 
\centerline{Sujet A - HA 1}
\begin{flushleft}
  \textbf{Partie cours} 
\end{flushleft} 
\begin{enumerate}
  \item Donner la définition de valeur absolue.
  \item Montrer que: pour tout réel x, on a $\sqrt{x^2} = |x|$.
\end{enumerate}
\textbf{Partie exercices}
\begin{enumerate}
  \item Montrer que: $A \Rightarrow B$ si et seulement si  $non (B) \Rightarrow non(A)$.
  \item Résolvez les équations suivantes (sans utiliser le $\Delta$). Soit x un nombre réel.
  \begin{enumerate}
    \item $x^2 - 3x + 2 = 0$
    \item $x^3 + 2x^2 + x = 0$
    \item $x^4 - 16 = 0$
    \item $|x-8| = 5$
    \item $\sqrt{x+2} = 3x+1$
  \end{enumerate}
  \item Dire si les assertions suivantes sont vraies ou fausses et donner leur négation:
  \begin{enumerate}
    \item $\exists x \in \mathbb{R} \;\; \forall y \in \mathbb{R} \;\; x + y > 0$
    \item $\forall x \in \mathbb{R} \;\; \exists y \in \mathbb{R} \;\; x+y > 0$
    \item $\forall x \in \mathbb{R} \; \; \forall y \in \mathbb{R} \;\; x + y > 0$
    \item $\exists x \in \mathbb{R} \;\; \forall y \in \mathbb{R} \;\; y^2 > x$
  \end{enumerate}
  \item Montrer par contraposition les assertions suivantes:
  \begin{enumerate}
    \item $\forall A, B \in \mathcal{P}(E) \; (A \cap B = A \cup B) \Rightarrow A = B$
    \item $\forall A, B, C \in \mathcal{P}(E) \; (A \cap B = A \cap C \; et \; A \cup B  = A \cup C) \Rightarrow B = C$
  \end{enumerate}
\end{enumerate}
\centerline{$\mathcal{MR}$}

\vfill\null
\columnbreak
% *****************
%  Sujet numéro 2
% *****************
\centerline{Sujet B - HA 1}

\begin{flushleft}
  \textbf{Partie cours} 
\end{flushleft} 
\begin{enumerate}
  \item Donner la définition d'une fonction majorée.
  \item Montrer pour tout (x; y) appartenant au cercle unité de $\mathbb{R}^2$, on $\{x; y\} \subset \left[-1;1\right]$
\end{enumerate}
\textbf{Partie exercices}
\begin{enumerate}
  \item Écrire la négation des assertions suivantes où P, Q, R, S sont des propositions:
  \begin{enumerate}
    \item $P \Rightarrow Q$
    \item P ET NON Q
    \item P ET (Q ET R)
    \item P OU (Q ET R)
    \item (P ET Q) $\Rightarrow (R \Rightarrow S)$ 
  \end{enumerate}
  \item Résolvez les équations et inéquations suivantes (sans utiliser le $\Delta$). Soit x un nombre réel.
  \begin{enumerate}
    \item $x^2 + x + 1 = 0$
    \item $x^3 + x^2 + x = 0$
    \item $|x-3| < 5$
    \item $\sqrt{3x-5} = x+1$
    \item $|x - 1| < \alpha$ où $\alpha$ est un paramètre réel strictement positif.
  \end{enumerate}
  \item Déterminer m pour que l'équation suivante ait deux racines réelles positives: $m^2x^2 + (m-3)x + 4 = 0$
  \item Montrer que $A \cap B = A \cap C \Leftrightarrow A \cap \complement{B} = A \cap \complement{C}$
\end{enumerate}
\centerline{$\mathcal{MR}$}

\vfill\null
\columnbreak
% *****************
%  Sujet numéro 3
% *****************
\centerline{Sujet C - HA 1}
\begin{flushleft}
  \textbf{Partie cours} 
\end{flushleft} 
\begin{enumerate}
  \item Ennoncer rigoureusement les deux façons pour décrire un ensemble .
  \item Démontrer que $\sqrt{2}$ est irrationnel.
\end{enumerate}
\textbf{Partie exercices}
\begin{enumerate}
  \item Soit m et p deux paramètres réels fixés. Résolvez les équations et innéquations suivantes en x (réel). Pensez à bien déterminer l'ensemble solution:
  \begin{enumerate}
    \item $(x+m)(x-p) > 0$
    \item $(mx-1)(4x-p) = 0$
    \item $(x+m)^2 = x+p$
    \item $x+m = \sqrt{x+p}$
    \item $|x-12| < m$
  \end{enumerate}
  \item La proposition $(P \wedge Q) \Rightarrow (\lnot P) \vee Q)$ est-elle vraie? 
  \item Montrer que $\forall \epsilon > 0 \;\; \exists N \in \mathbb{N}$ tel que $(n \geq N \Rightarrow 2 - \epsilon < \frac{2n+1}{n+2} < 2+\epsilon)$
  \item Écrire la négation des phrases suivantes:
  \begin{enumerate}
    \item $\forall x \in \mathbb{R} \; \exists n \in  \mathbb{N} \; | \; x \leq n$
    \item $\forall x \in \mathbb{R} \; \forall y \in \mathbb{R} \; xy = yx$
  \end{enumerate}
\end{enumerate}
\centerline{$\mathcal{MR}$}


\end{multicols*}
\end{document}
