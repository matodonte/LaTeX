\documentclass[a4paper,11pt, landscape]{article}

% Package pour la présentation (police, colonnes, marges ...)
\usepackage{ae,lmodern}
\usepackage[utf8]{inputenc}
\usepackage{multicol}
\usepackage[top=0.4cm, bottom=0.4cm, left=0.4cm, right=0.4cm]{geometry}
\usepackage{enumitem}

% Package pour les maths
\usepackage{amsthm}
\usepackage{amsmath}
\usepackage{amssymb}
\usepackage{mathrsfs}
\usepackage{amsfonts}
\usepackage{systeme}
%\usepackage{stmaryrd}

\begin{document}
\begin{multicols*}{3}
\setlength{\columnsep}{1cm}
% *****************
%  Sujet numéro 1
% ***************** 
\centerline{Sujet A - HA 8}
\begin{flushleft}
% *****************
%  Cours
% ***************** 
  \textbf{Partie cours} 
\end{flushleft} 

\begin{enumerate}[leftmargin=*]
  \item Voir lien envoyé: QCM en ligne.
  \item Soit $z \in \mathbb{C}$. Comparer $|z|$, $Im (z)$, et $|Im (z)|$. Démontrez et donnez un exemple.
\end{enumerate}
% *****************
%  Exercices
% ***************** 
\textbf{Partie exercices}
\begin{enumerate}[leftmargin=*]
  \item Soit $n \in \mathbb{N}^*$ et $(b_k)_{k \in [| 1; n|]} \in \mathbb{R}^n$
  \begin{enumerate}
    \item Montrez que $\sum\limits_{k=1}^{n}|b_k| \geq \left|\sum\limits_{k=1}^{n}b_k \right|$
    \item Déterminez une CNS pour que $\sum\limits_{k=1}^{n}|b_k| = \left|\sum\limits_{k=1}^{n}b_k \right|$
  \end{enumerate}
  \item Montrez que $\forall n \in \mathbb{N}^*$, $\sum\limits_{k=1}^{n}\frac{1}{k} \leq 2 \sqrt{n}$ 
  \item On considère l'équation dans $\mathbb{C}: X^3+pX+q = 0$ (*) où $(p;q)\in \mathbb{R}^2$.
  \begin{enumerate}
    \item Soit $x$ une racine de (*), on pose :$\systeme{u+v=x,uv=-\frac{p}{3}}$
    Justifier l'existence du couple $(u,v)$.\\ \\
    Montrer que $u^3$ et $v^3$ sont racines de l'équation $X^2 + qX-\frac{p^3}{27} = 0$.
    \item Réciproquement, soient $X'$ et$X''$ les racines de $X^2+qX-\frac{p^3}{27} = 0$. Montrer que l'on peut trouver une racine cubique $u$ de $X'$ et une racine cubique $v$ de $X''$ telles que $uv = -\frac{p}{3}$.\\
    En déduire les racines de (*).
    \item Discuter le nombre de racines réelles de (*).
    
  \end{enumerate}
  
\end{enumerate}
\centerline{$\mathcal{MR}$}

\vfill\null
\columnbreak
% *****************
%  Sujet numéro 2
% *****************
\centerline{Sujet B - HA 8}

\begin{flushleft}
% *****************
%  Cours
% ***************** 
  \textbf{Partie cours} 
\end{flushleft} 
\begin{enumerate}[leftmargin=*]
  \item Voir lien envoyé: QCM en ligne.
  \item Montrez l'existence de l'écriture d'un nombre complexe $z \in \mathbb{U}$ de module 1 sous la forme $e^{i\theta}$. Donnez une CNS sur $\theta \in \mathbb{R}$ et $\phi \in \mathbb{R}$ pour que $e^{i\theta} = e^{i\phi}$.
\end{enumerate}
% *****************
%  Exercices
% ***************** 
\textbf{Partie exercices}
\begin{enumerate}[leftmargin=*]
%------------
% Exercice 1
%------------
  \item Soit $n \in \mathbb{N}^*$. Pour $x \in \mathbb{R}$, on souhaite calculer: $S_n(x) := \sum\limits_{k=1}^{n}kx^k$. Vous utiliserez les 2 méthodes.
  \begin{enumerate}
    \item Méthode 1: Posez $\varphi_n : x \mapsto \sum\limits_{k=1}^{n}x^k$. Pour $x \in \mathbb{R}$, calculez $\varphi_n'(x)$ puis exprimez $S_n(x)$ en fonction $\varphi_n'(x)$.
    \item Méthode 2: En décrochant le dernier terme de $S_{n+1}(x)$, exprimez $S_{n+1}(x)$ en fonction de $S_{n}(x)$. En décrochant le dernier terme de $S_{n+1}(x)$, exprimez, d'une autre façon, $S_{n+1}(x)$ en fonction de $S_{n}(x)$.\\
    Concluez.
  \end{enumerate}
%------------
% Exercice 2
%------------
  \item On pose $\omega = \sqrt{3}+i$
  \begin{enumerate}
    \item Déterminer les $n \in \mathbb{Z}$ tels que $\omega^n \in \mathbb{R}$.
    \item Déterminer les $n \in \mathbb{Z}$ tels que $\omega^n \in i\mathbb{R}$.
  \end{enumerate}
%------------
% Exercice 3
%------------
\item
\begin{enumerate}
  \item Soit $z \in \mathbb{C}$. Trouvez une CNS sur $z$ pour que $|z-i| = |z+i|$
  \item Pour $z \in \mathbb{C} \smallsetminus\{2i\}$, posons $f(z) = \frac{z+2+3i}{z-2i}$. Quel est l'ensemble des points dont l'affixe $z$ vérifie: 
  \begin{enumerate}
    \item $f(z) \in \mathbb{U}$
    \item $f(z) \in i\mathbb{R}$
  \end{enumerate}
\end{enumerate}
%------------
% Exercice 4
%------------

\end{enumerate}
\centerline{$\mathcal{MR}$}
\vfill\null
\columnbreak
% *****************
%  Sujet numéro 3
% ***************** 
\centerline{Sujet C - HA 8}
\begin{flushleft}
% *****************
%  Cours
% ***************** 
  \textbf{Partie cours} 
\end{flushleft} 
\begin{enumerate}[leftmargin=*]
  \item Voir lien envoyé: QCM en ligne.
  \item Démontrez que l'ensemble des solutions $\mathcal{S}$ de l'équation $z^2-sz+p$ vérifie: $\mathcal{S} = \{u;v\} \Leftrightarrow (u+v = s$ et $uv = p)$.
\end{enumerate}
% *****************
%  Exercices
% ***************** 
\textbf{Partie exercices}
\begin{enumerate}[leftmargin=*]
%------------
% Exercice 1
%------------
  \item Soit $n \in \mathbb{N}$ et $\theta \in \mathbb{R}$.
  \begin{enumerate}
    \item Simplifiez la somme $D_n(\theta) = \sum\limits_{k=-n}^{n}e^{ik\theta}$ (Noyau de Dirichlet)
    \item Simplifiez la somme $F_n(\theta) = \frac{1}{n+1} \sum\limits_{k=0}^{n}D_k(\theta)$ (Noyau de Fejer)
  \end{enumerate}
%----------
% Exercice 2
%-----------
  \item Soit $n \in \mathbb{N}$. Calculez $T_n := \sum\limits_{k=1}^{n}\frac{1}{2^k}cos\left(\frac{k\pi}{3}\right)$
%-----------
% Exercice 3
%-----------
  \item On note $j$ le nombre complexe $e^{i\frac{2\pi}{3}}$.
  \begin{enumerate}
    \item Représentez graphiquement le point d'affixe $j$.
    \item Montrez que $j$ est solution de $z^3 = 1$. En déduire que $j$ puis $\bar{j}$ sont solutions de $1+z+z^2 = 0$.
    \item Comparez: $j$; $\frac{1}{j}$ ; $\frac{1}{j^2}$;$\bar{j}$; $\bar{j^2}$ ; $j^2$
    \item Que vaut $j^k$ pour que $k \in \mathbb{Z}$ ?
    En déduire la valeur de $1+j^k+j^{2k}$.
    \item Soit $n$ un entier naturel. Développez l'expression $(1+1)^n + (1+j)^n+(1+j^2)^n$
    \item En déduire la valeur de $\sum\limits_{0 \leq 3k \leq n}\binom{n}{3k}$
  \end{enumerate}
%-----------
% Exercice 4
%-----------
\item 
\end{enumerate}
\centerline{$\mathcal{MR}$}


\end{multicols*}
\end{document}
